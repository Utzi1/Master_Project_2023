\documentclass{beamer}
%\usetheme{boxes}
%\usecolortheme{rose}
\usepackage{amsmath}
\usepackage[T1]{fontenc} 
\usepackage[scaled=0.83]{sourcecodepro}
\usepackage[]{graphicx}
\usepackage{algpseudocode}
\usepackage[]{neuralnetwork}
\begin{tikzpicture}
	\begin{pgfonlayer}{nodelayer}
		\node [style=new style 0] (0) at (-12.5, -0.25) {Input};
		\node [style=new style 0] (2) at (-5.25, -0.25) {Encoder};
		\node [style=new style 0] (3) at (5.5, -1) {Var};
		\node [style=new style 0] (4) at (12.5, -2.5) {Decoder};
		\node [style=new style 0] (5) at (3, -2.5) {Sampling};
		\node [style=new style 0] (6) at (12.5, 1.5) {KL};
		\node [style=new style 0] (8) at (5.75, 0.5) {Mean};
		\node [style=new style 1] (9) at (6.75, -2.5) {$*$};
		\node [style=new style 1] (10) at (8.25, -2.5) {+};
		\node [style=new style 0] (11) at (22.5, -0.5) {Output};
	\end{pgfonlayer}
	\begin{pgfonlayer}{edgelayer}
		\draw [style=new edge style 0] (4) to (11);
		\draw [style=new edge style 0] (0) to (2);
		\draw [style=new edge style 0] (2) to (8);
		\draw [style=new edge style 0] (2) to (3);
		\draw [style=new edge style 0] (5) to (4);
		\draw [style=new edge style 0] (3) to (9);
		\draw [style=new edge style 0] (8) to (10);
		\draw [style=new edge style 0] (8) to (6);
		\draw [style=new edge style 0] (3) to (6);
	\end{pgfonlayer}
\end{tikzpicture}

\usepackage{amsxtra} 
\usepackage{tikzit}
\usepackage[]{cite}
%Information to be included in the title page:
\title{Halvtid presentation}
\author{ULR}
\institute{HIS}
\titlegraphic{\includegraphics[height=2.5cm]{Högskolan_i_Skövde_vapen.svg.png}}
\date{\today}

\begin{document}

\frame{\titlepage}
\begin{frame}{Contents}
    \tableofcontents
\end{frame}


\section{Autoencoder}
\begin{frame}{Autoencoder}
    \begin{block}{Basics}
        \begin{itemize}
            \item "been part of the historical landscape of of neural networks for decades"\cite{Goodfellow-et-al-2016}
            \item Hidden layers $h$ reassemble a function $f(X)$ which transform their input $X$ to a encoded
                representation $z$ so that another reconstruction function $g(z)$ is able to produce $g(z) \approx X$.
        \end{itemize}    
    \end{block}
    \begin{block}{Usage}
        \begin{itemize}
            \item Denoising
            \item Dimensionality reduction
            \item Feature learning
            \item Generative learning
        \end{itemize}
    \end{block}
\end{frame}

\begin{frame}{Autoencoder}
    \resizebox{\textwidth}{!}{	\begin{neuralnetwork}[height=6]
		\newcommand{\nodetextclear}[2]{}
		\newcommand{\nodetextx}[2]{$$}
		\newcommand{\nodetexty}[2]{$$}
		\inputlayer[count=6, bias=false, title=Input\\layer, text=\nodetextx]
		\inputlayer[count=3, bias=false, title=Encoder\\layer, text=\nodetextx]\linklayers
		\hiddenlayer[count=2, bias=false, title=Latent\\layer, text=\nodetextclear] \linklayers
		\outputlayer[count=3, title=Decoder\\layer, text=\nodetexty] \linklayers
		\outputlayer[count=6, title=Output\\layer, text=\nodetexty] \linklayers
	\end{neuralnetwork}
}
\end{frame}

\section{Variational Autoencoder}
\subsection{Background}

\begin{frame}{Variational Autoencoder}
    \begin{block}{Method}
        Traditionally unsupervised variational Bayesian method
    \end{block}
    \begin{columns}
        \begin{column}{0.30\textwidth}
            \begin{block}{Sampling process}
                \begin{center}
                    \resizebox{\textwidth}{!}{\tikzfig{VAEmodel}}
                \end{center}
            \end{block}
        \end{column}
        \begin{column}{0.65\textwidth}
            \begin{block}{}
                \begin{description}
                    \item[$\Theta$] Model parameter space
                    \item[$\rho$] Probability distribution $P(z)$
                    \item[$z$] Latent variable sampled $n$ times
                \end{description}
            \end{block}
        \end{column}
    \end{columns}
\end{frame}

\subsection{Mathematical background}

\begin{frame}{Variational Autoencoder}
    \begin{block}{Bayes theorem}
        \begin{equation}
            P(A\mid B) = \frac{P(B \mid A) P(A)}{P(B)}
        \end{equation}
    \end{block}

    \begin{block}{Objective}
        \begin{itemize}
            \item With every example $X$ from a set $\mathcal{X}$ we aim to generate a space $z$ to
                recreate $X$.
            \item $z$ gets sampled from $\mathcal{Z}$ according to $\rho \sim P(z)$.
            \item $f(z;\theta), f: \mathcal{Z} \times \Theta \to \mathcal{X}$ helps to optimize 
                $\theta$ 
            \item maximize:
                \begin{equation}
                    P(x) = \int P(X\mid z; \theta)\, P(z) \, dz
                \end{equation}
            \item Where $f(z;\theta)$ gets replaced by $P(X\mid z; \theta)$ to enable maximum 
                likelihood framework
        \end{itemize}
    \end{block}
\end{frame}


\begin{frame}{Variational Autoencoder}
    \begin{block}{Objective}
        Based on some input sample values of $z$ likely to produce $X$.
    \end{block}
    \begin{block}{Method}
        Therefor a function $Q(z\mid X)$ is introduced to generate $z$'s likely to produce $X$.

        This space of $z$'s should be smaller than that one under the prior $P(X)$
    \end{block}
\end{frame}

\begin{frame}{Variational Autoencoder}
    \begin{block}{Optimisation}
        The framework introduced by Kingma and Welling \cite{1312.6114} does the magic.
        It allows us to construct a differentiable estimator:
        \begin{equation}
            \int Q(z\mid X)\, f(z) \, dz
        \end{equation}
    \end{block}
    \begin{block}{Kullback Leibler Divergence}
        Allows to compare probability distributions
        \begin{description}
            \item[$P, Q$] probability distribution functions
        \end{description}
        \begin{equation}
            \mathcal{D}(P\parallel Q) = \mathcal{KL} = \sum_{x \in X} P(x) \cdot \log \frac{P(x)}{Q(x)}
        \end{equation}
    \end{block}
\end{frame}

\begin{frame}{Helpers}
    
    \begin{block}{Marginal likelihood of individual data points}
        (Likelihood function integrated over parameter space)
        \begin{equation}
            \log P\left(X^i\right) = \mathcal{D}\left( Q_\phi \left(z \mid X^i \right) \parallel P_\theta \left(z \mid
                X^i\right) \right) +\mathcal{L}  \left(\theta , \phi , X^i\right)
        \end{equation}
        Where:
        \begin{equation}
                \mathcal{L} (\theta , \phi , X) =
                \mathcal{E}_{z\sim Q} \left[\log P_\theta(z\mid X)-\log Q_\phi(z) \right]
        \end{equation}
    \end{block}
    \begin{block}{Evidence lower bound}
        \begin{equation}
            \begin{split}
                \log P_\theta \left(X^i\right) & \geq \mathcal{L} \left(\theta , \phi , X^i\right) \\
                                               &= \mathcal{E}_{Q_\phi (z\mid X)} \left[ -\log 
                                               Q_\phi(z \mid X) + \log P_\theta(z, X) \right]
            \end{split}
        \end{equation}
    \end{block}
\end{frame}

\begin{frame}{Relationship between $Q$ and $P$}
    The Evidence lower bound (ELBO) or just lower variational bound:
    \begin{equation}
        \mathcal{D} \left( Q(z) \parallel P(z\mid X) \right) = \mathcal{E}_{z\sim Q} \left[ \log Q(z) 
        - \log P(z\mid X) \right]
    \end{equation}
    Introduce $P(x)$ and $P(X\mid z)$:
    \begin{equation}
        \begin{split}
            \mathcal{D} & \left( Q(z) \parallel P(z\mid X) \right) \\
                        &= \mathcal{E}_{z\sim Q} \left[ \log Q(z) - 
                        \log P(z\mid X) -\log P(z) \right] + \log P(X)
        \end{split}
    \end{equation}
    Rearrange the ELBO and get another $\mathcal{KL}$-term:
    \begin{equation}
        \begin{split}
            \log P(X) - & \mathcal{D} \left( Q(z) \parallel P(z\mid X) \right) \\
                        &= \mathcal{E}_{z\sim Q} \left[\log P(X\mid z) \right] - \mathcal{D}
                        \left( Q(z) \parallel P(z) \right) 
        \end{split}
    \end{equation}
    And introduce a dependency of $Q(z)$ on $X$ to make it useful:
    \begin{equation}
        \begin{split}
            \log P(X) - & \mathcal{D} \left( Q(z\mid X) \parallel P(z\mid X) \right) \\
                        &= \mathcal{E}_{z\sim Q} \left[\log P(X\mid z) \right] - \mathcal{D}
                        \left( Q(z \mid X) \parallel P(z) \right)
        \end{split}
    \end{equation}
\end{frame}

\begin{frame}{Autoencoding variational Bayes}

    \alert{Backpropagation only works with continuous operations, stochastic units aren't that.}
    \begin{block}{The reparametrisation trick}
        With $\mu(X)$ and $\sigma(x)$ parametrising $Q(z \mid X)$ sampling from 
        $\mathcal{N}(\mu(X), \sigma(X))$ we can draw $\epsilon \sim \mathcal{N}(0, 1)$ and compute 
        $z_{\epsilon \sim Q} = \mu(X) + \sqrt{\sigma}(X) \cdot \epsilon$ and perform SGD

        \begin{equation}
            \mathcal{E}_{z\sim \mathcal{D}} \left[\mathcal{E}_{\epsilon \sim \mathcal{N}(0,1)} 
                \left[\log P(X\mid z_{\epsilon \sim Q} - \mathcal{D}
                \left( Q(z \mid X) \parallel P(z) \right)
\right] \right]
        \end{equation}
        \begin{equation}
            \begin{split}
                \mathcal{L} (\theta , \phi , X) &=
                \mathcal{E}_{z\sim Q} \left[\log P_\theta(z\mid X)-\log Q_\phi(z) \right]\\
                                                &\approx \tilde{\mathcal{L}}^m(\theta , \phi , X^m) 
                                                = \frac{n}{m} \sum_{i=1}^{m} \mathcal{L} (\theta , \phi , X_i)
            \end{split}
        \end{equation}

    \end{block}

\end{frame}


\begin{frame}{}
    \begin{block}{Pseudocode}
        \begin{algorithmic}
            \State $\theta, \phi \gets $ random parameter
            \Repeat
            \State $X^m$ $\gets$ random mini batch with size $m$
            \State $\epsilon$ $\gets$ random samples from noise distribution $P(\epsilon)$
            \State $g$ $\gets$ $\nabla _{\theta , \phi} \tilde{\mathcal{L}}^m(\theta , \phi , X^m , \epsilon) $
            \State $\theta, \phi \gets $ update parameters with $g$
            \Until convergence of params $\theta, \phi$
            \State \Return $\theta, \phi$
        \end{algorithmic}
    \end{block}
\end{frame}

\begin{frame}{Variational Autoencoder}
    \resizebox{\textwidth}{!}{\tikzfig{VAE}}
    \resizebox{\textwidth}{!}{	\begin{neuralnetwork}[]
        \newcommand{\nodeEnc}[2]{E}
        \newcommand{\nodeDec}[2]{D}
        \inputlayer[count=10, bias=false]
        \hiddenlayer[count=10, bias=false, text=\nodeEnc]\linklayers
        \hiddenlayer[count=5, bias=false, text=\nodeEnc]\linklayers
        \hiddenlayer[count=3, bias=false, text=\nodeEnc]\linklayers
            %\link[from layer=3, to layer=4, from node=1, to node=1]
        \hiddenlayer[count=4, bias=false, title=Sampling]\linklayers
        \hiddenlayer[count=2, bias=false, title=Latent space]\linklayers
        \hiddenlayer[count=5, bias=false, text=\nodeDec]\linklayers
        \hiddenlayer[count=4, bias=false, text=\nodeDec]\linklayers
        \hiddenlayer[count=10, bias=false, text=\nodeDec]\linklayers
        \outputlayer[count=10, bias=false]\linklayers
    \end{neuralnetwork}

}
\end{frame}

\subsection{Implementation}

\begin{frame}{Implementation}

    \begin{itemize}
        \item Using \texttt{tensorflow.keras} functional API \cite{Chollet2017-yw}
        \item Currently $relu$ activations and unbiased layers
        \item Hidden layers constructed iterative
        \item Deploying a \texttt{.\_build} method, returns encoder, decoder and z
        \item \texttt{Model} gets trained in customised step
        \item Sampling layer forms an extra \texttt{Class}
    \end{itemize}

\end{frame}

\begin{frame}[fragile]{Training step}

    \tiny{
        \begin{verbatim}
 def train_step(self, data):                                                
    with tf.GradientTape() as tape:                                        
        z_mean, z_logvar, z = self.encoder(data)                           
        reconstruction = self.decoder(z)                                   
        reconstruction_loss = mean(K.square(data - reconstruction),      
                                     axis=1)                               
        kl_loss = -.5 * (1 + z_logvar - square(z_mean) - exp(          
            z_logvar)                                                      
                         )                                                 
        kl_loss = sum(kl_loss)                                           
        total_loss = kl_loss + reconstruction_loss                         
    grads = tape.gradient(total_loss, self.trainable_weights)              
    self.optimizer.apply_gradients(zip(grads, self.trainable_weights))     
    self.total_loss_tracker.update_state(total_loss)                       
    self.reconstruction_loss_tracker.update_state(reconstruction_loss)     
    self.kl_loss_tracker.update_state(kl_loss)                             
    return {                                                               
            "loss": self.total_loss_tracker.result(),                      
            "reconstruction_loss": self.reconstruction_loss_tracker.result(),
            "kl_loss": self.kl_loss_tracker.result()}
        \end{verbatim}
    }
\end{frame}

\begin{frame}[fragile]{Sampling layer}

    \tiny{
        \begin{verbatim}
class Sampling_Layer(tf.keras.layers.Layer):                                 
    """                                                                      
    A sampling layer using the mean and log(var) to sample from an input     
    """                                                                      
    def call(self, inputs):                                                  
        mean, logvar = inputs                                                
        batch = tf.shape(mean)[0]                                            
        dim = tf.shape(mean)[1]                                              
        epsilon = tf.random.normal(shape=(batch, dim), mean=0., stddev=1.)   
        return mean + tf.exp(logvar * .5) * epsilon                          
        \end{verbatim}
    }
\end{frame}

\begin{frame}[fragile]{Training with a dummy}
    \tiny{
        \begin{verbatim}
import tensorflow as tf                                                       
import numpy as np                                                            
import generic_VAE                                                            

# create some random, uniformly distributed tensors:                          
training_dummys = np.asarray(                                                 
        [tf.random.uniform((400,)) for item in range(100000)])                

# create some testing tensors:                                                
testing_dummys = np.asarray(                                                  
        [tf.random.uniform((400,)) for item in range(50000)])                 

# this is completely senseless, the part where a training-set and one for     
# testing was created the VAE is unsupervised (in this case)                  
dummy_data = np.concatenate((training_dummys, testing_dummys))                

dummy_vae = generic_VAE.Builder(                                              
        input_shape=(400, ),                                                  
        encoder_shape=[400, 200, 100, 40, 20, 4],                             
        decoder_shape=[4, 20, 40, 100, 200, 400],                             
        latent_dims=2,                                                        
        dropout_rate=0)                                                       

vae = generic_VAE.VAE(dummy_vae.decoder_model, dummy_vae.encoder_model)       
vae.compile()                                                                 
vae.fit(dummy_data, epochs=10, batch_size=2000)                               
        \end{verbatim}
    } 
\end{frame}    

\begin{frame}[fragile]{Training with a dummy}
    \Tiny{
        \begin{verbatim}
        Epoch 1/10
        75/75 [==============================] - 4s 32ms/step - loss: 3.6552 - reconstruction_loss: 0.1831 - kl_loss: 3.4721
        Epoch 2/10
        75/75 [==============================] - 3s 34ms/step - loss: 0.1436 - reconstruction_loss: 0.1432 - kl_loss: 4.0093e-04
        Epoch 3/10
        75/75 [==============================] - 3s 38ms/step - loss: 0.1430 - reconstruction_loss: 0.1422 - kl_loss: 7.2372e-04
        Epoch 4/10
        75/75 [==============================] - 3s 33ms/step - loss: 0.1417 - reconstruction_loss: 0.1410 - kl_loss: 7.1526e-04
        Epoch 5/10
        75/75 [==============================] - 3s 34ms/step - loss: 0.1405 - reconstruction_loss: 0.1398 - kl_loss: 7.1526e-04
        Epoch 6/10
        75/75 [==============================] - 3s 35ms/step - loss: 0.1403 - reconstruction_loss: 0.1396 - kl_loss: 7.1526e-04
        Epoch 7/10
        75/75 [==============================] - 3s 42ms/step - loss: 0.1397 - reconstruction_loss: 0.1389 - kl_loss: 7.1526e-04
        Epoch 8/10
        75/75 [==============================] - 3s 36ms/step - loss: 0.1395 - reconstruction_loss: 0.1388 - kl_loss: 7.1526e-04
        Epoch 9/10
        75/75 [==============================] - 3s 35ms/step - loss: 0.1395 - reconstruction_loss: 0.1388 - kl_loss: 7.1526e-04
        Epoch 10/10
        75/75 [==============================] - 3s 33ms/step - loss: 0.1391 - reconstruction_loss: 0.1384 - kl_loss: 7.1526e-04
                \end{verbatim}
            } 
\end{frame}    


\section{Training data}

\begin{frame}{Training with GED}

    \cite{Wang2018-wk}
\end{frame}

\section{Next steps}

\begin{frame}{Next step}
    \cite{Choi2021-gp} \cite{}
\end{frame}

\section{Bibliography}
\begin{frame}[allowframebreaks]{Bibliography}

    \bibliography{biblio}{}
    \bibliographystyle{plain}

\end{frame}
\end{document}
